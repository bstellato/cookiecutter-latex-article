\documentclass[12pt]{article}

% Images and plots
\usepackage{graphicx}
\usepackage{pgfplots}

% Math
\usepackage{amsmath,amssymb,amsthm,enumitem,mathtools}
\usepackage{multirow}
\usepackage{algorithm}% http://ctan.org/pkg/algorithms
\usepackage[noend]{algpseudocode} % Avoid "end" and make it look cleaner

% Tables
\usepackage{csvsimple}	% reading CSV files in tables
\usepackage{booktabs}   % Nicer Tables
\usepackage{adjustbox}  % To adjust table length

% Hyperreferences
\usepackage{hyperref}

% Acronyms
\usepackage[xindy, acronyms]{glossaries}   % Glossaries and Acronyms
\makeglossaries

% Todos
\usepackage{todonotes}
\newcommand{\mynote}[1]{\todo[inline, caption={}]{ {{cookiecutter.author}} : #1}}
\newcommand{\todofigure}[0]{\missingfigure[figwidth=\textwidth, figcolor=white]{} }

% Definitions
% Commands
\newcommand{\eg}{{\it e.g.}}
\newcommand{\ie}{{\it i.e.}}

% Sets
\newcommand{\ones}{\mathbf 1}
\newcommand{\reals}{{\mbox{\bf R}}}
\newcommand{\integers}{{\mbox{\bf Z}}}
\newcommand{\symm}{{\mbox{\bf S}}}  % symmetric matrices
\newcommand{\NPhard}{\mbox{$\mathcal{NP}$-hard}}  % symmetric matrices

\newcommand{\prob}{{\mathbf P}}
\newcommand{\distrib}{{\mathcal P}}
\newcommand{\identity}{I}
\newcommand{\tpose}{T}
\newcommand{\nullspace}{{\mathcal N}}
\newcommand{\range}{{\mathcal R}}
\newcommand{\Rank}{\mathop{\bf Rank}}
\newcommand{\Tr}{\mathop{\bf Tr}}
\newcommand{\diag}{\mathop{\bf diag}}
\newcommand{\lambdamax}{{\lambda_{\rm max}}}
\newcommand{\lambdamin}{\lambda_{\rm min}}
\newcommand{\Expect}{\mathop{\bf E{}}}
\newcommand{\Prob}{\mathop{\bf Prob}}
\newcommand{\Co}{{\mathop {\bf Co}}} % convex hull
\newcommand{\dist}{\mathop{\bf dist{}}}
\newcommand{\argmin}{\mathop{\rm argmin}}
\newcommand{\argmax}{\mathop{\rm argmax}}
\newcommand{\epi}{\mathop{\bf epi}} % epigraph
\newcommand{\Vol}{\mathop{\bf vol}}
\newcommand{\dom}{\mathop{\bf dom}} % domain
\newcommand{\intr}{\mathop{\bf int}}

\newcommand{\sign}{\mathop{\bf sign}}
\newcommand{\round}{\mathop{\bf round}}
\newcommand{\define}{\coloneqq}

% Trees
\newcommand{\tree}{\mathcal{T}}

% Strategy
\newcommand{\strategy}{\mathcal{S}}
\newcommand{\intvars}{\mathcal{I}}
\newcommand{\tightconstraints}{\mathcal{T}}

% Machine learning
\newcommand{\loss}{\mathcal{L}}

% Sections (for citation)
\newcommand{\Sec}{Section\;}


% Create theorems and other environments
\newtheorem{theorem}{Theorem}[section]  % Restart counter every section
\newtheorem{lemma}{Lemma}[section]  % Restart counter every section
\newtheorem{corollary}{Corollary}[theorem]  % Restart counter every theorem
\newtheorem{proposition}{Proposition}[section]
\newtheorem{assumption}{Assumption}[section]
\newtheorem{definition}{Definition}[section]
\newtheorem{example}{Example}[section]
\newtheorem{exercise}{Exercise}[section]
\newtheorem{remark}{Remark}[section]
\newtheorem{problem}{Problem}[section]

% Acronyms
\newacronym{LP}{LP}{linear program}
\newacronym{QP}{QP}{quadratic program}
\newacronym{MIQP}{MIQP}{mixed-integer quadratic program}
\newacronym{MIO}{MIO}{mixed-integer optimization}
\newacronym{MILP}{MILP}{mixed-integer linear program}
\newacronym{MINLP}{MINLP}{mixed-integer nonlinear program}
\newacronym{SPM}{SPM}{successive projection method}
\newacronym{sBB}{sBB}{spacial branch and bound}
\newacronym{NLP}{NLP}{nonlinear program}
\newacronym{PWA}{PWA}{piecewise affine}
\newacronym{SVM}{SVM}{support vector machines}
\newacronym{OCT}{OCT}{optimal classification tree}
\newacronym{OCT-H}{OCT-H}{optimal classification trees with-hyperplanes}
\newacronym{CART}{CART}{classification and regression tree}
\newacronym{NN}{NN}{neural network}
\newacronym{ReLU}{ReLU}{rectified linear unit}
\newacronym{CPU}{CPU}{central processing unit}
\newacronym{GPU}{GPU}{graphics processing unit}
\newacronym{MPC}{MPC}{model predictive control}
\newacronym{ADMM}{ADMM}{alternating direction method of multipliers}
\newacronym{ADP}{ADP}{approximate dynamic programming}
\newacronym{FPGA}{FPGA}{field-programmable gate array}



% Bibliography
\bibliographystyle{alpha}


\title{ {{cookiecutter.title}} }
\author{ {{cookiecutter.author}} }

\begin{document}
\maketitle

\begin{abstract}
We present...
\end{abstract}

% Table of contents
%\newpage \tableofcontents \newpage


\section{Introduction}

Let's try one acronym,~\gls{LO}. Let's try one reference~\cite{conforti2014}.

\mynote{Here is a todo note.}


\subsection{Related work}

\section{Our method}

Here is an example of optimization problem
\begin{equation*}
	\begin{array}{ll}
		\text{minimize} & f(x)\\
		\text{subject to} & g(x)\le 0,
	\end{array}
\end{equation*}
where $x \in \reals^n$ is the optimization variable.
See {\tt definitions.tex} to get familiar and possibly add new math definitions.


\mynote{Here is another todo note.}

\todofigure

\section{Results}
Here is an example table.

\begin{table}
  \centering
  \caption{Example table.}
  \label{tab:tablelabel}
  \begin{tabular}{ll}
    \toprule
    $A$ & $B$\\
    \midrule
    % Add directly from csv reader
    % the csv file has names colnameA, colnameB
    \csvreader[head to column names, late after line=\\]{./data/data.csv}{
    colnameA=\colA,
    colnameB=\colB
    }{\colA & \colB}
    \bottomrule
  \end{tabular}
\end{table}

Figure~\ref{fig:tikz} show an example of how to use TikZ.
\begin{figure}
	\centering
  \begin{tikzpicture}
    \begin{axis}[
        samples=100, % you don't need 1000, it only slows things down
        ticks=none,
        xmin = -2.5, xmax = 2.5,
        ymin = -1, ymax = 3,
        axis x line=middle,
        axis y line=middle,
	xlabel={$x$},
        ylabel={$y$},
	x label style={
	  at={(axis cs:2,-0.2)},
	  anchor=west,
	},
	declare function={f(\x)=abs(\x);},
        ]
      \addplot[thick,color=black, mark=none, domain=-1:1, -,shorten >=1pt] {f(x)};
      \draw[dashed] (axis cs:-1,0) node[below=0.5mm] {$-1$} -- (axis cs:-1,2);
      \draw[thick] (axis cs:-1,1) -- (axis cs:-1,2);
      \draw[dashed] (axis cs:1,0) node[below=0.5mm] {$1$} -- (axis cs:1,2);
      \draw[thick] (axis cs:1,1) -- (axis cs:1,2);
      \addplot[mark=*,fill=black] coordinates {(0,0)};
      \addplot[mark=*,fill=black] coordinates {(1,1)};
      \addplot[mark=*,fill=black] coordinates {(-1,1)};
      \draw[->, thick](axis cs:2,2)--(axis cs:2.3,2.2) node[below=0.5mm](q){$\alpha$};
    \end{axis}
  \end{tikzpicture}
  \caption{Example of TikZ figure.}
  \label{fig:tikz}
  \end{figure}


% Bibliography
\bibliography{bibliography}

\end{document}
